\documentclass[]{article}
\usepackage{amsmath, amsthm, amsfonts}
\usepackage{graphics}
\usepackage{float}
\usepackage{epsfig}
\usepackage{amssymb}
\usepackage{dsfont}
\usepackage{latexsym}
\usepackage{newlfont}
\usepackage{epstopdf}
\usepackage{amsthm}
\usepackage{epsfig}
\usepackage{caption}
\usepackage{multirow}
\usepackage[pdftex,breaklinks,colorlinks,linkcolor=black,citecolor=blue,urlcolor=blue]{hyperref}
\usepackage[x11names,table]{xcolor}
\usepackage{graphics}
\usepackage{wrapfig}
\usepackage[rflt]{floatflt}
\usepackage{multicol}
\usepackage{listings} \lstset {language = C++, basicstyle=\bfseries\ttfamily, keywordstyle = \color{blue}, commentstyle = \bf\color{green}}
\usepackage{tikz}
\usepackage{enumitem}

\newcommand
*
{\itembolasazules}[1]{% bolas 3D
	\footnotesize\protect\tikz[baseline=-3pt]%
	\protect\node[scale=.5, circle, shade, ball
	color=blue]{\color{white}\Large\bf#1};}
%opening
\title{Proyecto de L\'ogica Difusa}
\author{Karl Lewis Sosa Justiz}

\begin{document}

\begin{figure}
	\maketitle
	\hspace{3,5cm} \includegraphics[width=5cm]{images/índice.jpg} 
\end{figure}


\clearpage
\tableofcontents
\newpage
\section{Propuesta de Problema}
Se quiere obtener la sensaci\'on t\'ermica en las personas tomando en cuenta la temperatura, la velocidad del viento y la humedad.
\section{Caracter\'isticas del Sistema De inferencia propuesto}
\subsection{Variables }

\begin{enumerate}
	\item Temperatura(en grados Celsius) que se clasifica:
	\begin{enumerate}
		\item frio (TrapezoidalFuzzyNumber(0, 0, 10,25))
		\item normal (TriangularFuzzyNumber(15, 25, 30))
		\item caliente (TrapezoidalFuzzyNumber(25, 30, 40,400))
	\end{enumerate}
	\item Humedad (Humedad Relativa)  que se clasifica:
	\begin{enumerate}
		\item baja (TrapezoidalFuzzyNumber(0, 0, 40,50))
		\item regular (TriangularFuzzyNumber(40, 55, 70))
		\item caliente (TrapezoidalFuzzyNumber(60, 70, 100,100))
	\end{enumerate}
	\item Viento (Velocidad del viento en KM/H)
		\begin{enumerate}
			\item baja (TrapezoidalFuzzyNumber(0, 0, 40,50))
			\item regular (TriangularFuzzyNumber(40, 55, 70))
			\item caliente (TrapezoidalFuzzyNumber(60, 70, 100,100))
		\end{enumerate}
		\item Sensacion(en grados Celsius) que se clasifica:
		\begin{enumerate}
			\item frio (TrapezoidalFuzzyNumber(0, 0, 10,25))
			\item normal (TriangularFuzzyNumber(15, 25, 30))
			\item caliente (TrapezoidalFuzzyNumber(25, 30, 40,400))
		\end{enumerate}
\end{enumerate}

\section{Reglas}
Son cargadas del archivo rules.txt y siguen la gram\'atica:
\subsection{Gram\'atica}
 program : list\_rules \newline   
 list\_rules : list\_rules rule \newline
 list\_rules : rule \newline
 rule : IF exp THEN list\_statement \newline
 list\_statement : list\_statement COMA statement \newline
 list\_statement : statement \newline
 exp : exp AND term \newline
 exp : exp OR term \newline
 exp : term \newline
 term : NOT term \newline
 term : OPAR exp CPAR \newline
 term : statement \newline
 statement : VAR IS ADJ
\subsection{Reglas del problema}
\begin{enumerate}
	\item IF Temperatura is frio AND  Humedad  is baja and Viento is calmado  then Sensacion is frio  
	\item IF Temperatura is frio and Humedad is alta AND Viento is calmado then Sensacion is normal
	\item if Temperatura is normal and Humedad is regular and Viento is calmado then Sensacion is normal
	\item if Temperatura is normal and Humedad is alta  and NOt Viento is intenso then Sensacion is caliente
	\item IF Temperatura is normal and (Viento is intenso or Viento is moderado) then Sensacion is frio
	\item IF Temperatura is caliente and Viento is intenso then Sensacion is normal
	\item If Temperatura is caliente AND (Viento is moderado or Viento is calmado) then Sensacion is caliente
\end{enumerate}
\section{Principales ideas seguidas para la implementaci\'on}


\textbf{RuleLexer:} Utilizado para tokenizar las reglas. \newline
\textbf{RuleParser:} Utilizado para parsear las reglas.  \newline
\textbf{LinguisticVar:} Forma gen\'erica para las variables que contiene tanto las categor\'ias como las funciones de pertenencia asociadas.\newline
\textbf{membership\_functions} Formas de representar las funciones de pertenencia triangular y trapezoidal.\newline
\textbf{FuzzySystem} Grueso de la implementaci\'on que contiene tanto como los m\'eto dos de agregaci\'on  (Mamdani, Larsen) desdifusificaci\'on. \newline
(Centroide, Bisecci\'on y Media de Ma\'ximos).\newline
\section{Requisitos}
Es necesario instalar ply en python, se encuentra en requirements.txt, en otro caso siempre se pueden armar las reglas a mano =(.
\section{Consideraciones}
Los Sistema de Inferencia Difusa, en particular el implementado, permite la soluci\'on de problemas a un nivel m\'as cercano al razonamiento humano lo cual es una mejor\'ia enorme, adema\'as permiten ir mejorando el conjunto de reglas para ir logrando cada vez mejores y m\'as reales resultados, de una manera m\'as sencilla e intuitiva. En el problema particular es com\'un encontrar tablas que dan resultados similares pero de esta forma se puede ajustar seg\'un la persona y de esta forma ser a\'un m\'as precisos a la hora de obtener pron\'osticos o por ejemplo que vestir seg\'un estos datos.  
\end{document}}